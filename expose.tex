\section{Kontext}\label{sec:kontext}

Im Unternehmen wurde eine Projektverwaltungssoftware entwickelt.
Diese hat unter anderem folgende Funktionen:
\begin{itemize}
  \item Für Kundenprojekte können Pläne angelegt werden.
  \item Mitarbeiter:innen können Stunden für die Projekte erfassen.
  \item Am Ende eines Monats können aus den erfassten Stunden die Rechnungen an die Kunden erstellt werden.
  \item Projektleiter:innen können die Stunden der Mitarbeiter:innen einsehen.
  \item Projektleiter:innen haben eine Übersicht über die Kosten und Umsätze der Projekte.
\end{itemize}

Das Backend ist in Java mit Spring Boot geschrieben und verwendet als
Datenspeicher eine MySQL Datenbank.
Das Frontend ist in Vue.js geschrieben.

Die Applikation ist im Moment eine self-hosted single tenant Applikation.
Deployt wird der Stack in Docker Containern auf einer virtuellen Maschine mithilfe von docker-compose.

Um die Applikation an Kunden verkaufen zu können, soll sie als Software
as a Service (SaaS) angeboten werden.
Das heißt, die Software kann über die Webseite von mehreren Mandaten genutzt werden.


\section{Problemstellung}\label{sec:problemstellung}

Im Moment ist die Applikation so aufgebaut, dass eine Instanz der Applikation nur von einem Mandanten genutzt werden kann.
Das heißt in der Datenbank, sind alle Tabellen so ausgelegt, dass die Daten von einem Mandanten kommen.
Im Backend wird immer auf dieselbe Datenbank zugegriffen und es gibt keine Möglichkeit zwischen mehreren Mandanten zu unterscheiden.
Im Frontend ist es nicht möglich sich als Mandant einen Zugang zu kaufen.

Dadurch entstehen folgende Probleme.
Im Frontend muss es die Möglichkeit geben, sich als Mandant, einen Zugang zu kaufen und einzurichten.
Im Backend muss diese Funktionalität unterstützt werden.
An einer Stelle zwischen Frontend und Datenbank muss der jeweilige Mandant festgestellt werden, sodass die entsprechenden Daten ausgewählt werden.
Hier ist die Schwierigkeit zu entscheiden, auf welcher Ebene die Logik und die Daten voneinander getrennt werden sollen.
Zum Beispiel, ob es pro Mandant eigene Datenbanken, eigene Tabellen geben soll oder nur über Attribute innerhalb einer Tabelle die Mandanten unterschieden werden sollen.

\section{Lösungsansatz}\label{sec:loesungsansatz}

Im Rahmen der Bachelorarbeit sollen verschiedene Lösungsansätze verglichen werden.
Dafür soll in der Literatur nach vergleichbaren Problemen und deren Lösungen recherchiert werden.
Um das Vergleichen zu ermöglichen, sollen Kriterien gefunden und definiert werden, anhand derer die Lösungen verglichen werden können.
Anhand der recherchierten Quellen sollen Lösungsansätze erstellt und mit den Kriterien verglichen werden.
Danach soll eine, für das Beispielprojekt am besten geeignete Lösung, als Prototyp umgesetzt werden.

\section{Vorgehen}\label{sec:vorgehen}

\begin{enumerate}
  \item Literaturrecherche
  \begin{itemize}
    \item Mithilfe der strukturierten Literaturrecherche sollen Quellen gefunden werden
  \end{itemize}
\item
  Finden und definieren der Kriterien anhand derer, die Lösungen möglichst objektiv
  miteinander verglichen werden können.

  \begin{itemize}
  \item
    Dabei soll ein Kriterium möglichst so definiert werden, dass es
    quantitativ bestimmbar ist.
  \item
    ist dies nicht möglich, soll eine Skala von 1 bis 10 verwendet
    werden.
  \end{itemize}
\item
  Verschiedene mögliche Lösungen sammeln

  \begin{itemize}
    \item Die Lösungen können aus der Literaturrecherche auf die Beispielapplikation passend, angepasst und kombiniert werden.
  \item
    Die Lösungen sollen möglichst weit und konkret ausgearbeitet werden.
  \item
    Hierfür sollen, soweit sinnvoll, auch UML-Diagramme erstellt werden.
  \end{itemize}
\item
  Mögliche Lösungen theoretisch miteinander vergleichen

  \begin{itemize}
  \item
    Jede Lösung soll anhand der Kriterien bewertet werden
  \item
    Dabei soll die Bewertung anhand der Beispielapplikation und dem
    gegebenem Kontext erfolgen.
    So soll sichergestellt werden, dass die
    Werte Vergleichbar sind.
  \end{itemize}
\item Eine Lösung für das Beispielprojekt auswählen

  \begin{itemize}
  \item
    Anhand der Kriterien soll die möglicherweise beste Lösung herausgesucht werden
  \end{itemize}
\item
  Lösung in einem Prototyp umsetzen

  \begin{itemize}
  \item
    Die rausgesuchte Lösung soll als Prototyp umgesetzt werden
  \item
    Während des Erstellens sollen Metriken gesammelt werden, die
    relevante Kriterien beschreiben

    \begin{itemize}
    \item
      Initial Aufwand (Arbeitsstunden)
    \item
      Notwendiges Wissen
    \end{itemize}
  \item
    Am Ende der Implementierung sollen Metriken für die anderen
    Kriterien gesammelt werden

    \begin{itemize}
    \item
      Wartungsaufwand: Wird wahrscheinlich eine Schätzung sein, aus der
      gesammelten Erfahrung
    \item
      Laufende Kosten: Zum einen die aktuell anfallenden Kosten.
      Zudem kann es eine Hochrechnung geben, wenn die Applikation skaliert.
    \item
      Skalierbarkeit: Hierfür sollen verschiedene Benchmarks durchgeführt werden
    \item
      (optional) Sicherheit: Hierfür kann ein Pentest durchgeführt werden
    \end{itemize}
  \end{itemize}
\item
  Auswertung

  \begin{itemize}
  \item
    Gesammelte Metriken mit zuvor berechneten Kriterien vergleichen
  \item
    Implementierung im allgemeinen auswerten
  \end{itemize}
\end{enumerate}

\section{Zeitplan}\label{sec:zeitplan}

Gesamt 16 Wochen (4 Monate a 4 Wochen)

\begin{itemize}
\item
  07.10: Anmeldung der BA
\item
  12.10: Problemstellung (0.5 Woche)
\item
  19.10: Kriterien festlegen und definieren (1 Woche)
\item
  23.10: Mögliche Lösungen entwerfen (2 Wochen)
\item
  06.11: Lösungen miteinander vergleichen (2 Wochen)
\item
  11.12: Implementierung (5 Wochen)
\item
  25.12: Evaluierung/Ergebnis und Ausblick (2 Wochen)
\item
  01.01: Puffer/Pause (1 Woche)
\item
  08.01: Grundlagen (1 Woche)
\item
  12.01: Einleitung (0.5 Woche)
\item
  19.01: Layout und Korrekturlesen (1 Woche)
\item
  26.01: Drucken (1 Woche)
\item
  27.01: Abgabe BA
\end{itemize}
