\section{Über die Firma}
Die Frachtwerk GmbH wurde im Jahr 2017 in Berlin als Unternehmensberatung mit Fokus auf Logistik und Transport gegründet.
Mittlerweile werden in einem zweiten Büro in Karlsruhe Softwareentwicklung und System Operations als weitere Dienstleistungen angeboten.
In der Firma arbeiten knapp 40 Mitarbeiter:innen.
Der größte Kunde ist die Deutsche Bahn AG mit unter anderem den Subunternehmen DB Netz AG, DB Bus AG und DB Regio AG.

\section{Kontext}\label{sec:kontext}

Im Unternehmen wurde eine Projektverwaltungssoftware entwickelt.
Diese hat unter anderem folgende Funktionen:
\begin{itemize}
  \item Für Kundenprojekte können Pläne angelegt werden. Ein Plan beschreibt unter anderem, wer wie viele Stunden mit welchen Kosten an einem Projekt arbeiten kann. 
  \item Mitarbeiter:innen können Stunden für die Projekte erfassen.
  \item Am Ende eines Monats können aus den erfassten Stunden die Rechnungen an die Kunden erstellt werden.
  \item Projektleiter:innen können die Stunden der Mitarbeiter:innen einsehen.
  \item Projektleiter:innen haben eine Übersicht über die Kosten und Umsätze der Projekte.
\end{itemize}

Das Backend ist in Java mit Spring Boot geschrieben und verwendet als
Datenspeicher eine MySQL Datenbank.
Das Frontend ist in Vue.js geschrieben.

Die Applikation ist im Moment eine self-hosted single tenant Applikation.
Das heißt, die Applikation wird auf einer virtuellen Maschine für einen einzelnen Mandanten deployt.
In dem Fall für das eigene Unternehmen.
Deployt wird der Stack in Docker Containern auf einer virtuellen Maschine mithilfe von docker-compose.

Um die Applikation an Kunden verkaufen zu können, soll sie als Software
as a Service (SaaS) angeboten werden.
Das heißt unter anderem, dass die Software über die Webseite von mehreren Mandanten genutzt werden kann.


\section{Problemstellung}\label{sec:problemstellung}

Im Moment ist die Applikation so aufgebaut, dass eine Instanz der Applikation nur von einem Mandanten genutzt werden kann.
Das heißt in der Datenbank sind alle Tabellen so ausgelegt, dass die Daten von einem Mandanten kommen.
Im Backend wird immer auf dieselbe Datenbank zugegriffen und es gibt keine Möglichkeit zwischen mehreren Mandanten zu unterscheiden.
Im Frontend ist es nicht möglich, sich als Mandant einen Zugang zu kaufen.

Dadurch entstehen folgende Anforderungen.
Im Frontend muss es die Möglichkeit geben, sich als Mandant, einen Zugang zu kaufen und einzurichten.
Im Backend muss diese Funktionalität unterstützt werden.
An einer Stelle zwischen Frontend und Datenbank muss der jeweilige Mandant festgestellt werden, sodass ein Mandant nur Zugriff auf seine eigenen Daten hat.
Hier ist die Schwierigkeit zu entscheiden, auf welcher Ebene die Logik und die Daten voneinander getrennt werden sollen.
Zum Beispiel, ob es pro Mandant eine eigene Datenbank, eigene Tabellen geben soll oder nur über Attribute innerhalb einer Tabelle die Mandanten unterschieden werden sollen.

Zu den oben genannten Anforderungen kommen noch nicht funktionale Anforderungen wie Skalierbarkeit, Betriebskosten, Anpassbarkeit etc. dazu.

\section{Lösungsansatz}\label{sec:loesungsansatz}

Zu Beginn der Bachelorarbeit sollen durch Literaturrecherche Eigenschaften gesammelt und vorgestellt werden, die eine Software as a Service Applikation ausmacht.
In einer weiteren Literaturrecherche sollen dann Lösungsansätze recherchiert und vorgestellt werden, um diese Eigenschaften zu erfüllen.
Um das Vergleichen verschiedener Lösungen zu ermöglichen, sollen Kriterien gefunden und definiert werden, anhand derer die Lösungen verglichen werden können.
Die herausgearbeiteten Lösungen sollen anhand der Kriterien bewertet werden.
Danach soll eine, für das Beispielprojekt am besten geeignete Lösung, als Prototyp umgesetzt werden.
Das durch die Implementierung entstandene Ergebnis soll mit der theoretischen Lösung verglichen werden.
Am Ende soll ein Fazit gezogen werden.

\section{Vorgehen}\label{sec:vorgehen}

\begin{enumerate}
  \item Literaturrecherche
  \begin{itemize}
    \item Mithilfe der strukturierten Literaturrecherche sollen Quellen gefunden werden
    \item Zuerst sollen Eigenschaften gesammelt werden, die eine SaaS ausmachen.
    \item Danach sollen Lösungen für diese Eigenschaften recherchiert werden.
  \end{itemize}
\item
  Finden und definieren der Kriterien anhand derer, die Lösungen möglichst objektiv
  miteinander verglichen werden können.

  \begin{itemize}
  \item
    Dabei soll ein Kriterium möglichst so definiert werden, dass es
    quantitativ bestimmbar ist.
  \item
    Qualitative Merkmale sollen in eine Skala von 1 bis 10 übersetzt werden.
  \end{itemize}
\item
  Verschiedene mögliche Lösungen ausarbeiten

  \begin{itemize}
    \item Für die notwendigen Eigenschaften, Lösungen ausarbeiten
  \item
    Die Lösungen sollen so weit und konkret ausgearbeitet werden, dass ein Vergleich möglich ist.
  \end{itemize}
\item
  Mögliche Lösungen theoretisch miteinander vergleichen

  \begin{itemize}
  \item
    Jede Lösung soll anhand der Kriterien bewertet werden
  \item
    Dabei soll die Bewertung anhand der Beispielapplikation und dem
    gegebenem Kontext erfolgen.
    So soll sichergestellt werden, dass die
    Werte vergleichbar sind.
  \end{itemize}
\item Eine Lösung für das Beispielprojekt auswählen

  \begin{itemize}
  \item
    Anhand der Kriterien soll die für diesen Kontext beste Lösung herausgesucht werden
    \item Bei der Auswahl soll eine Lösung gefunden werden, die gut abgeschnitten hat und in der vorgegebenen Zeit umsetzbar ist.
  \end{itemize}
\item
  Lösung in einem Prototyp umsetzen

  \begin{itemize}
  \item Während der Entwicklung sollen soweit sinnvoll automatisierte Tests erstellt werden.
  \item Über eine CI/CD Pipeline sollen die Änderungen regelmäßig getestet und deployt werden.
  \item
    Während und am Ende der Implementierung, sollen Metriken gesammelt werden, um die zuvor erarbeiteten Kriterien auszuwerten.
  \end{itemize}
\item
  Auswertung

  \begin{itemize}
  \item
    Gesammelte Metriken mit zuvor berechneten Kriterien vergleichen
  \item
    Implementierung im Allgemeinen auswerten
  \end{itemize}
  \item Fazit
\begin{itemize}
    \item Die implementierte Lösung soll bewertet werden.
    \item Die Übertragbarkeit auf andere Projekte soll beschrieben werden.
    \item In einem Ausblick soll dargestellt werden, wie es mit der Applikation weiter geht.
\end{itemize}
\end{enumerate}

\section{Zeitplan}\label{sec:zeitplan}

Gesamt 16 Wochen (4 Monate a 4 Wochen)

\begin{itemize}
\item
  14.10: Anmeldung der BA
\item
  19.10: Problemstellung (0.5 Wochen)
    \item 19. 10: Literaturrecherche und Ergebnisse sammeln (1 Woche)
    \item 26.10: Ergebnisse verschriftlichen (1 Woche)
  \item 02.11: Kriterien festlegen und definieren (1 Woche)
\item
  23.11: Mögliche Lösungen entwerfen (2 Wochen)
\item
  30.11: Lösungen miteinander vergleichen (1 Woche)
\item
  21.11: Implementierung (3 Wochen)
\item
  04.01: Fazit (2 Wochen)
\item
  11.01: Grundlagen fertigstellen (1 Woche)
\item
  15.01: Einleitung (0.5 Woche)
\item
  22.01: Layout und Korrekturlesen (1 Woche)
\item
  25.01: Drucken (0.5 Wochen)
  \item Puffer (0.5 Wochen)
\item
  27.01: Abgabe
\end{itemize}

